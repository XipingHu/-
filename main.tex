\documentclass{article}
\usepackage{amsmath}
\usepackage{amsfonts}
\usepackage{amssymb}
\usepackage{ctex}
\usepackage{graphicx}
\usepackage{float}
\usepackage{geometry}
\geometry{a4paper,scale=0.8}
\author{Xiping Hu}
\title{积分中三角换元的一点问题}
\begin{document}
\maketitle
\noindent \textbf{问题} \quad 为什么积分的时候可以做三角变换?三角函数值域不能覆盖R域,我们怎么知道在1之外这个变换是正确的? by fht
\vspace{1cm}
\newline
\noindent 
\textbf{解答}\quad
考虑积分
\begin{equation}
  \label{}
\int {\frac {dx}{\sqrt {a^{2}-x^{2}}}}
\end{equation}
我们令
\begin{equation}
  \label{}
x = a \sin \theta
\end{equation}
则有
\begin{equation}
  \label{}
dx=a\cos \theta d \theta 
\end{equation}
\begin{equation}
  \label{}
\theta = \arcsin \left( \frac{x}{a} \right )
\end{equation}
则有
\begin{equation}
  \label{}
\begin{aligned}\int {\frac {dx}{\sqrt {a^{2}-x^{2}}}}&=\int {\frac {a\cos \theta \,d\theta }{\sqrt {a^{2}-a^{2}\sin ^{2}\theta }}}\\&=\int {\frac {a\cos \theta \,d\theta }{\sqrt {a^{2}(1-\sin ^{2}\theta )}}}\\&=\int {\frac {a\cos \theta \,d\theta }{\sqrt {a^{2}\cos ^{2}\theta }}}\\&=\int d\theta \\&=\theta +C\\&=\arcsin \left({\frac {x}{a}}\right)+C.\end{aligned}
\end{equation}
\par 其中在这个函数中$x$的定义域为$[-a.a]$,$\theta$的定义域为$[-\frac{\pi}{2} , \frac{\pi}{2}]$
当$\theta = -\frac{\pi}{2}$时,$\sin \theta=-1$,$x$正好在边界$x=-a$上。
当$\theta = \frac{\pi}{2}$时,$\sin \theta=1$,$x$正好在边界$x=a$上。
这没有任何定义域冲突或者不符合的问题。
\end{document}\frac{numerator}{denominator}


